\documentclass{pkupcl}
\title{某物理量的测定}
\author{张三}
\id{1800000000}
\group{第1组1号}
\date{2020 年 1 月 1 日}
\temperature{20.0}
\pressure{100.0}
\abstract{本实验测定了某物理量。}
\keyword{甲\quad 乙\quad 丙\quad 丁}
\begin{document}
\makecover
\section{引言}

一些引言。

\section{方法}

\subsection{仪器和试剂}

\subsubsection{试剂}

\ce{H2O}。

\subsubsection{仪器}

\subsection{实验内容}

\begin{enumerate}
	\item ……;
	\item ……。
\end{enumerate}

\section{结果与讨论}

\subsection{主要结果 1 及其讨论}

数据如表 1 所示。

\begin{table}[!ht]
\centering
\caption{表格}	
\begin{tabular}{cc}
\toprule
表头 A & 表头 B \\ \midrule
1     & 2      \\
3     & 4      \\ \bottomrule
\end{tabular}
\end{table}

\subsection{主要结果 2 及其讨论}

图片如图 1 所示。

\begin{figure}[!ht]
\begin{center}
\includegraphics[width=0.7\textwidth]{logo.png}
\caption{图片}
\end{center}
\end{figure}

\section{实验结论}

\section*{参考文献}

\begin{enumerate}[label={[\arabic*]}]
	\item 北京大学化学学院物理化学实验教研组,物理化学实验(第4版),北京大学出版社,2002.
	\item 松鼠,Chem \& Coding | 用LaTeX完成物化实验报告(2) 封面的制作,知乎,2017.
\end{enumerate}

\end{document}
